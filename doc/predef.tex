

\usepackage{amsmath}
\usepackage{amssymb}
\usepackage{amsthm}
\usepackage{amsfonts}
\usepackage{graphicx}
\usepackage{mathtools}
\usepackage{hyperref}
\usepackage{xspace}
\usepackage{aliascnt}

\newcommand{\lfalse}{\textbf{false}\xspace}
\newcommand{\ltrue}{\textbf{true}\xspace}
\newcommand{\Int}{\mathbb{Z}}
\newcommand{\IntMod}[1]{\Int_{#1}}
\newcommand{\unique}{\exists!}
\newcommand{\defeq}{\mathrel{\mathop:}=}
\newcommand{\acteq}{\longrightarrow}
\newcommand{\actto}{\longmapsto}
\renewcommand{\mod}{{}\textnormal{\texttt{\;mod\;}}{}}
\newcommand{\modop}[1]{\textnormal{\texttt{\;mod\;}}#1}

\newcommand{\textiff}{\textit{iff}\xspace}
\newcommand{\quicksec}[1]{\noindent{\bf #1.}\;}
\newcommand{\funcname}[1]{\textsc{#1}}
\newcommand{\funcall}[2]{\funcname{#1}(#2)}
\newcommand{\vbl}[1]{\textnormal{\textit{#1}}}
\newcommand{\ttvbl}[1]{\texttt{#1}}
\newcommand{\ttaref}[2]{\ttvbl{#1[}#2\ttvbl{]}}
\newcommand{\ttmemloc}[1]{\texttt{\&#1}}
\newcommand{\mathmemloc}[1]{\ensuremath{\texttt{\&}#1}}

\DeclarePairedDelimiter{\abs}{\lvert}{\rvert}
\DeclarePairedDelimiter{\ceil}{\lceil}{\rceil}
\DeclarePairedDelimiter{\floor}{\lfloor}{\rfloor}
\DeclarePairedDelimiter{\avect}{\langle}{\rangle}
\DeclarePairedDelimiter{\aset}{\allowbreak\lbrace}{\rbrace}

\newcommand{\lxor}{\veebar}
\newcommand{\true}{\top}
\newcommand{\fals}{\bot}
\newcommand{\ket}[1]{\lvert #1\rangle}
%\newcommand{\Nat}{\mathbb{N}}
\newcommand{\WildSym}{\texttt{X}\xspace}
\newcommand{\transpose}{^ \top }
\newcommand{\queseq}{\overset{?}{=}}
%\newcommand{\transpose}{^ \intercal }
\newcommand{\pipe}{\,|\,}
\newcommand{\detop}[1]{\det(#1)}
%\newcommand{\vect}[1]{\boldsymbol{\vbl{#1}}}
\newcommand{\vect}[1]{{\overrightarrow{#1}}}
%\newcommand{\slfrac}[2]{\left.#1\middle/#2\right.}
\newcommand{\slfrac}[2]{#1/#2}
%\newcommand{\vect}[1]{{\vv{#1}}}
\newcommand{\fmlang}[1]{\textnormal{\textrm{#1}}}
\providecommand{\expten}[1]{\ensuremath{\times 10^{#1}}}


\def\imagetop#1{\vtop{\null\hbox{#1}}}
\newcommand{\TODO}[1]{\noindent\textcolor{red}{{\bf TODO:} #1}}
\newcommand{\NOTE}[1]{\noindent\textcolor{red}{{\bf NOTE:} #1}}
\newcommand{\QUES}[1]{\noindent\textcolor{red}{{\bf ???:} #1}}

\mathchardef\breakingcomma\mathcode`\,
{\catcode`,=\active
 \gdef,{\breakingcomma\discretionary{}{}{}}
}
\newcommand{\mathlist}[1]{\mathcode`\,=\string"8000 #1}

\newcommand{\quickcol}[2]{%
\begin{tabular}{@{}#1@{}}#2\end{tabular}}

\makeatletter
\@ifclassloaded{beamer}{
%%%%%%% BEG In beamer
\usetheme{Warsaw}
\setbeamertemplate{headline}{}
\setbeamertemplate{bibliography item}[text]
\setbeamertemplate{navigation symbols}{%
 \begin{tabular}{r}%
 \insertframenavigationsymbol \\
 {\scriptsize \color{gray} \insertframenumber{}/\inserttotalframenumber{}}
 \end{tabular}%
}
%%%%%%% END In beamer
}{
%%%%%%% BEG Not in beamer
\usepackage{paralist}

\newenvironment{itemize*}%
{\begin{compactitem}}%
{\end{compactitem}}

\newenvironment{enumerate*}%
{\begin{compactenum}}%
{\end{compactenum}}

\@ifundefined{ifusesection}{%
 \newif\ifusesection %
 \usesectiontrue %
}{}

\ifusesection
 \newtheorem{theorem}{Theorem}[section]
\else
 \newtheorem{theorem}{Theorem}
\fi

\def\sectionautorefname{Section}
%\def\subsectionautorefname{Subsection}
\def\subsectionautorefname{Section}
\def\subsubsectionautorefname{Section}
\def\chapterautorefname{Chapter}

% counter for numbering, and make them work with \autoref.
\newcommand{\mynewtheorem}[2]{
 \newaliascnt{#1}{theorem}
 \newtheorem{#1}[#1]{#2}
 \aliascntresetthe{#1}
 % maybe we will squish some autoref defaults, but who cares?
 \expandafter\def\csname #1autorefname\endcsname{#2}
}
\mynewtheorem{lemma}{Lemma}
\mynewtheorem{corollary}{Corollary}
\mynewtheorem{example}{Example}
\theoremstyle{definition}
\mynewtheorem{problem}{Problem}
\mynewtheorem{definition}{Definition}
%\theoremstyle{remark}
\mynewtheorem{remark}{Remark}
\mynewtheorem{observation}{Observation}

%%% hyperref link stuff
\hypersetup{
 pdfborderstyle={/S/U/W 0.5}
}
%%%%%%% END Not in beamer
}
\makeatother

\newcounter{exercise}
\def\theexercise{\arabic{exercise}}
\newenvironment{exercise}[1][]%
{\allowbreak\begin{samepage}\vspace*{2em}\hrule%
 \refstepcounter{exercise}
 {\bf Exercise \theexercise.}\ifx\newenvironment#1\newenvironment\else\space(#1)\fi\\}
{\vspace*{0.5em}\hrule\end{samepage}\vspace*{1em}}


\newcounter{exercisepart}[exercise]
\def\theexercisepart{\theexercise.\alph{exercisepart}}
\newenvironment{exercisepart}[1][]%
{\allowbreak\begin{samepage}\refstepcounter{exercisepart}
 {\bf Exercise \theexercisepart.}\ifx\newenvironment#1\newenvironment\else\space(#1)\fi\space}
{\end{samepage}

}

\def\exerciseautorefname{Exercise}
\def\exercisepartautorefname{Exercise}

\usepackage{algorithm}
\usepackage{algorithmic}

\providecommand\algorithmname{Algorithm}

\newcommand{\algorithmicoutput}{\textbf{Output:}}
\newcommand{\OUTPUT}{\item[\algorithmicoutput]}

\newcommand{\LET}{\STATE \textbf{let}\xspace}


